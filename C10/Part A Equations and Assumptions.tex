\documentclass[12pt]{article}
\usepackage{amsmath}
\usepackage{amssymb}
\usepackage{graphicx}
\usepackage{hyperref}
\usepackage[latin1]{inputenc}
\usepackage[landscape]{geometry}

\makeatletter
\DeclareRobustCommand{\volume}{\text{\volumedash}V}
\newcommand{\volumedash}{%
	\makebox[0pt][l]{%
		\ooalign{\hfil\hphantom{$\m@th V$}\hfil\cr\kern0.08em--\hfil\cr}%
	}%
}
\makeatother

\begin{document}
	\hfill\break\break
	\textbf{Note:} \hfill\break
	\textbf{-}The size of each square was measured to be $20.1mm\pm0.05mm$. The size of several squares on each image was measured in pixels and averaged to create a scaling factor between pixels and millimeters.
	\hfill\break
	\textbf{-}The volumetric flow rate through the system \textit{for all experiments} was measured to be $\dot{\volume}=2.1 \frac{\textrm{gal}}{\textrm{min}}=132,000 \frac{mm^{3}}{s}$
	\hfill\break
	\textbf{-}The cylinder was measured to be $60.6mm\pm0.05mm$ in diameter.
	
	\hfill\break\break
	\textbf{Free-Stream Velocity:}
	Assuming an incompressible fluid, the free-stream velocity, at the inlet to the system, can be expressed as:
	$$U_{\infty}=\frac{\dot{\volume}}{A_{c}}$$
	where, ${A_{c}}$ is the entire cross-sectional area between the glass plane and the base, perpendicular to the horizontal plane. \hfill\break
	Since the plane was observed to be 24 squares wide and the glass was separated from the base surface by the thickness of the acrylic plates at $t_{plate}=3.2mm\pm0.05mm$, the free-stream velocity is:
	$$U_{\infty}=\frac{132,000\frac{mm^{3}}{s}}{24*20.1mm * 3.2mm}=85.9\frac{mm}{s}$$
	
	\hfill\break\break
	\textbf{Streamline Fundamentals:}
	Equation 4 of the assignment provides an equation for the determining the volumetric flow rate per unit depth between between two stream lines $\psi_{2}$ and $\psi_{1}$:
	$$\dot{\volume}'=q=\psi_{2}-\psi_{1}$$
	where $\psi$ is a constant value defining a given stream line which can be determined from models of the flow through an environment (such as around a cylinder), and $\psi_{2}$ has a greater $y$ value than $\psi_{1}$.
	
	\hfill\break\break
	\textbf{Flow Velocities}:
	Based on the above relations, assuming an incompressible fluid, the average flow velocity between two stream lines $\psi_{2}$ and $\psi_{1}$ can be determined as:
	$$U_{1,2}= \frac{\dot{\volume}'}{y_2-y_1}=\frac{\psi_{2}-\psi_{1}}{y_2-y_1}$$
	
	\hfill\break\break
	\textbf{Cylinder Streamlines:}
	Assuming inviscid flow ($\nabla^{2}\varphi=0$) and irrotational flow ($\nabla\times\vec{V}=0 \rightarrow \nabla^{2}\psi=0$) allowing for superposition of both potential and streamline equations, the assignment's equation 8 gives the equation for a streamline for flow around a submerged cylinder:
	$$ \psi = U_{\infty}sin(\theta)\left(r-\frac{a^{2}}{r}\right)$$
	where $\psi$ is a constant value defining a given stream line, $U_{\infty}$ is the free-stream flow velocity, $a$ is the radius of the cylinder, and ($r,$ $\theta$) are the polar coordinates of points along a streamline.
	\hfill\break\break
	By definition, the equation for the streamline can be used to determine orthogonal components of the flow velocity. For the submerged cylinder, the assignment gives these as: 
	$$u_{r}=\frac{1}{r}\left[\frac{\partial\psi}{\partial\theta}\right]=U_{\infty}cos(\theta)\left(1-\frac{a^2}{r^2}\right),
	\;\;u_{\theta}=-\left[\frac{\partial\psi}{\partial\theta}\right]=-U_{\infty}sin(\theta)\left(1+\frac{a^2}{r^2}\right)$$.
	\hfill\break
	Accordingly, since $u_r$ and $u_\theta$ are orthogonal, the magnitude of the flow velocity along a streamline at a given point $(r,\theta)$ can be given by:
	$$U(r,\theta)=\sqrt{u_r^2+u_\theta^2}=U_\infty\sqrt{\frac{\left(a^4-2\,\cos\left(2\,\theta\right)\,a^2\,r^2+r^4\right)}{r^4}}$$
\end{document}
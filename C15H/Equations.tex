\documentclass[12pt]{article}
\usepackage[table,dvipsnames]{xcolor}
\usepackage{amsmath}
\usepackage{amssymb}
\usepackage{graphicx}
\usepackage{hyperref}
\usepackage[latin1]{inputenc}
\usepackage[landscape]{geometry}
\usepackage{caption}

\definecolor{lightgrey}{gray}{0.9}

\makeatletter
\DeclareRobustCommand{\volume}{\text{\volumedash}V}
\newcommand{\volumedash}{%
	\makebox[0pt][l]{%
		\ooalign{\hfil\hphantom{$\m@th V$}\hfil\cr\kern0.08em--\hfil\cr}%
	}%
}
\makeatother

\renewcommand*{\thesubsection}{\alph{subsection}.}

\begin{document}
	\section{Part D Background and Theory}
	\hfill\break\break
	\subsection{Symbol Directory:}
	{\rowcolors{2}{white}{lightgrey}
	\centering
	\begin{tabular}{|c|l|c|}
		\hline
		\multicolumn{3}{|c|}{\textbf{Experimental Quantities:}} \\
		\hline
		\hline
		$y$ & Position Perpendicular to Plate & $mm$ \\
		$x$ & Position Along Plate from Leading Edge & $mm$ \\
		$x_c$ & Centerstream X-Position & $8.0mm$ \\
		$x_d$ & Downstream X-Position & $12.6mm$ \\
		$\delta x$ & Uncertainty in X-Position Measurements & $0.05mm$ \\
		$h_s$ & Static Pressure Head & $m$\\
		$h_p$ & Pitot Tube Pressure-Head & $m$\\
		$T_{amb}$ & Ambient Temperature & $22.8^{\circ}C$\\
		$\rho_a$ & Density of Ambient Air & $^{kg}/_{m^3}$ \\
		$U_{\infty}$ & Measured Free-Stream Velocity & $^m/_s$ \\
		\hline
	\end{tabular}
	}
	\hfill\break\break\break
	{\rowcolors{2}{white}{lightgrey}
	\centering
	\begin{tabular}{|c|l|c|}
		\hline
		\multicolumn{3}{|c|}{\textbf{Computed Quantities:}} \\
		\hline
		\hline
		$\delta \textbf{X}$ & Notation for Uncertainty in Some Measurement "\textbf{X}" & ${}$\\
		$V_{\infty rel}$ & Max. Flow Velocity Reached in a Measurement Cross-Section & $^m/_s$ \\
		$\Delta h$ & Difference in Static and Pitot Pressure Heads & $m$\\
		$P_{dyn}$ & Dynamic Pressure & $Pa$\\
		$V_c$ & Computed Flow Velocity at Pitot Tube Opening & $^m/_s$\\
		$Re$ & Reynolds Number & ${}$ \\
		$\delta$ & Boundary Layer Thickness & $mm$ \\
		$a,b,c$ & Velocity Curve Regression Parameters & ${}$ \\
		$SE_V$ & Standard Error of Velocity Regression & ${^m/_s}$ \\
		\hline
	\end{tabular}
	}

	\hfill\break\break
	\begin{table}
		\centering
		{\rowcolors{2}{white}{lightgrey}
		\begin{tabular}{|c|l|c|}
		\hline
			\multicolumn{3}{|c|}{\textbf{Constants and Thermophysical Properties:}} \\
		\hline
		\hline
			g & Acceleration due to Gravity & $9.807\frac{m}{s^2}$\\
			$\rho_f$ & Pitot Tube Fluid (water) Density & $^*997\frac{kg}{m^3}$\\
			$\nu_a$ & Kinematic Viscosity of Air & $^*15.36\times10^{-6}\frac{m^2}{s}$\\
		\hline
		\end{tabular}
		}
		\caption*{\textit{*Value taken from \em{Principles of Heat and Mass Transfer by Incropera, et al.} Textbook at Ambient Temperature}}
	\end{table}


	\subsection{Basic Calculations:}
	\paragraph{Uncertainty} For each set of measurements requested by the assignment, 10 were taken so that the uncertainty caused by both inherent uncertainty in the instrumentation and environmental factors could be captured by the standard deviation in the measurements.\hfill\break 
	For some parameter $\chi$ which was directly measured $n=10$ times, the value, $X$, used to represent this parameter throughout subsequent calculations was determined as follows:
	\begin{equation}
		X = \frac{1}{n}\sum_{i=1}^{n}{\chi_i}
	\end{equation}
	$$\bar{\chi} = \frac{1}{n}\sum_{i=1}^{n}{\chi_i}$$
	$$\sigma_X = \sqrt{\frac{\sum_{i=1}^{n}{(\chi_i-\bar{\chi})^2}}{n-1}}$$
	For 95\% confidence:
	\begin{equation}
		\delta X = 2\sigma_X
	\end{equation}
	
	
	\subsection{Analysis Calculations:}
	*All uncertainties of derived values were determined by standard error propagation techniques.
	\paragraph{Free-Stream Velocity} In the FAQ of the assignment, it stated that it is important for Reynolds number and boundary layer calculations to use the free-stream velocity \textit{in the cross-section where the measurements were taken} rather than the velocity measured upstream of the experiment provided by the data-logs, $U_\infty$.\hfill\break
	As such, for these calculations the free-stream velocity was assumed to be the maximum of all calculated velocities for the experimental setup in question (ie, same plate, set speed, pitot tube x-position).
	\begin{equation}
		V_{\infty rel_i} = max\{V_c\}_i
	\end{equation}
	\paragraph{Reynolds Number} Reynolds number was computed using standard technique for flow over a flat plate:
	\begin{equation}
		Re_x = \frac{\rho_a V_{\infty rel} x}{\nu_a}
	\end{equation}
	
	\paragraph{Calculated Experimental Boundary Layer Thickness} The method of least squares was employed. By deriving the relationship from equation 20, 21, and 22 of the "Best-Fit Polynomial Approximation Handout", the following was regression functions were determined:
	\hfill\break
	For laminar flow:
	\begin{equation}
		V(y) = a*y^3 + b*y
	\end{equation}
	where
	$$ A = \sum_{i=1}^{n}{y_i^6}$$
	$$ B = \sum_{i=1}^{n}{y_i^4}$$
	$$ C = \sum_{i=1}^{n}{V_{c_i} y_i^3}$$
	$$ D = \sum_{i=1}^{n}{y_i^4}$$
	$$ E = \sum_{i=1}^{n}{y_i^2}$$
	$$ F = \sum_{i=1}^{n}{V_{c_i} y_i}$$
	\begin{equation}
		a = \frac{CE- FB}{AE - BD}
	\end{equation}
	\begin{equation}
	a = \frac{AF - CD}{AE - BD}
	\end{equation}
	\hfill\break\break
	For turbulent flow:
	\begin{equation}
	V(y) = cy^{1/7}
	\end{equation}
	where
	\begin{equation}
	c = \frac{\sum_{i=1}^{n}{V_{c_i} y_i^{1/7}}}{\sum_{i=1}^{n}{y_i^{2/7}}}
	\end{equation}
	\hfill\break\break
	With a regression completed, in both cases the boundary layer is determined by MATLAB to solve the definition of a boundary layer given in the relevant portion of the assignment:
	\begin{equation}
		V(\delta) = 0.95V_{\infty rel}
	\end{equation}
	
	\hfill\break\break
	In both cases, uncertainties of the regression models were calculated using a 95\% confidence interval based on the standard error:
	
	\begin{equation}
	\delta \delta = 1.96SE = 1.96\sqrt{\frac{\sum_{i=1}^{n}{V(y_i) - \frac{1}{n}\sum_{i=1}^{n}{V_{c_i}}}}{n-2}}
	\end{equation}
	
	
	\paragraph{Theoretical Boundary Layer Thickness}
	For laminar flow, theoretical boundary layer thickness was determined using equation 18 of the assignment:
	
	\hfill\break
	For turbulent flow, equation 19 of the assignment:
	\hfill\break\break
	For turbulent flow which starts at the leading edge (smooth plate with obstacle), equation 20 of the assignment:
	
	\paragraph{Choosing Which Model to Use:} For the smooth plate, whether to use a laminar or turbulent flow model was determined strictly by the Reynolds number (as given above). For the rough plate, both the laminar and turbulent flow curves were regressed and which ever had the lower standard error was used. For the smooth plate with the obstacle, the flow was always assumed to be turbulent, starting at the leading edge.
	
	
\end{document}
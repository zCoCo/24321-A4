\documentclass[12pt]{article}
\usepackage[table,dvipsnames]{xcolor}
\usepackage{amsmath}
\usepackage{amssymb}
\usepackage{graphicx}
\usepackage{hyperref}
\usepackage[latin1]{inputenc}
\usepackage[landscape]{geometry}
\usepackage{caption}

\definecolor{lightgrey}{gray}{0.9}

\makeatletter
\DeclareRobustCommand{\volume}{\text{\volumedash}V}
\newcommand{\volumedash}{%
	\makebox[0pt][l]{%
		\ooalign{\hfil\hphantom{$\m@th V$}\hfil\cr\kern0.08em--\hfil\cr}%
	}%
}
\makeatother

\renewcommand*{\thesubsection}{\alph{subsection}.}

\begin{document}
	\section{Part D Background and Theory}
	\hfill\break\break
	\subsection{Symbol Directory:}
	{\rowcolors{2}{white}{lightgrey}
	\centering
	\begin{tabular}{|c|l|c|}
		\hline
		\multicolumn{3}{|c|}{\textbf{Experimental Quantities:}} \\
		\hline
		\hline
		$Q_m$ & Measured Volumetric Flow Rate & $\frac{L}{s}$\\
		$T$ & Fluid Temperature & $^{\circ}C$ \\
		$h_1$ & Upstream Head & $m$ \\
		$d_1$ & Upstream Pipe Diameter & $m$ \\
		$A_1$ & Upstream Cross Sectional Area & $m^2$ \\
		$h_0$ & Orifice/Venturi Throat Head & $m$ \\
		$d_0$ & Orifice/Venturi Throat Diameter & $m$ \\
		$A_0$ & Orifice/Venturi Throat Cross Sectional Area & $m^2$ \\
		\hline
	\end{tabular}
	}
	\hfill\break\break\break
	{\rowcolors{2}{white}{lightgrey}
	\centering
	\begin{tabular}{|c|l|c|}
		\hline
		\multicolumn{3}{|c|}{\textbf{Computed Quantities:}} \\
		\hline
		\hline
		$V_{av}$ & Average Fluid Velocity & $\frac{m}{s}$\\
		$[Re]$ & Reynolds Number & ${}$ \\
		$\Delta h$ & Empirical Minor Head Loss & $m$\\
		$Q_X$ & Calculated Volumetric Flow Rate for X & $\frac{L}{s}$\\
		$Q_{mX}$ & Relative Difference between Calculated Volumetric Flow Rate X and the Measured Value & $\frac{L}{s}$\\
		\hline
	\end{tabular}
	}

	\hfill\break\break
	\begin{table}
		\centering
		{\rowcolors{2}{white}{lightgrey}
		\begin{tabular}{|c|l|c|}
		\hline
			\multicolumn{3}{|c|}{\textbf{Constants and Thermophysical Properties:}} \\
		\hline
		\hline
			g & Acceleration due to Gravity & $9.807\frac{m}{s^2}$\\
			$\rho$ & Fluid Density & $\frac{kg}{m^3}$\\
			$\mu$ & Fluid Dynamic Viscosity & $\frac{N}{m^2s}$\\
			$\nu$ & Fluid Kinematic Viscosity & $\frac{m^2}{s}$\\
			$C_d$ & Discharge Coefficient & ${}$\\
		\hline
		\end{tabular}
		}
		\caption*{\textit{*All non-standard values extracted from relevant sections of assignment. \\ Values not provided were either not needed or were determined as \\ a function of some other variable.}}
	\end{table}


	\subsection{Basic Calculations:}
	\paragraph{Average Velocity}
		Assuming an incompressible fluid, for flow through a given tube, the average velocity, $V_{av}$ at any cross-section can be determined from the volumetric flow rate, $Q$, anywhere in the stream and the diameter of the tube, $d$ at the cross-section of interest as:
		\begin{equation}
			V_{av} = \frac{Q}{A_{c}} = \frac{4Q}{\pi d^2}
		\end{equation}
		
	\paragraph{Reynolds Number}
	As given in equation 8 of the assignment, the Reynolds Number can be determined by:
	\begin{equation}
		[Re] = \frac{\rho V_{av}d}{\mu} = \frac{V_{av}d}{\nu}
	\end{equation}
	
	\paragraph{Head Loss}
	By definition, head loss can be computed as:
	\begin{equation}
 		\Delta h = h_1 - h_2
	\end{equation}
	
	\subsection{System Calculations:}
	\paragraph{Head Loss and Flow Rate Measurement:}
	If the head loss over a certain very short region which experiences a change in cross sectional area is known, the Bernoulli and Venturi principles can be applied to measure the volumetric flow rate through that area as outline in equation 11 of the assignment:
	\begin{equation}
		Q = C_dA_0\sqrt{\frac{2g(\Delta h)}{1-(\frac{A_0}{A_1})^2}} = \frac{C_d\pi d_0^2}{4}\sqrt{\frac{2g(\Delta h)}{1-(\frac{d_0}{d_1})^4}}
	\end{equation}
	where $C_d$ is the discharge coefficient, $\Delta h$ is the head loss, and $A_0$ and $A_1$ are the cross sectional areas of the throat of the Venturi tube or Orifice and upstream pipe respectively.
	\paragraph{Uncertainties:}
	Per the specific request of prompt D.2, the uncertainty in calculated flow rates, $Q$, due to uncertainties in the measurements of $d_0$ and $d_1$, can be determined by combining partial uncertainties of the equation for $Q$ given in equation 11 of the assignment as follows:
	$$\delta Q^2 = \left[\frac{\partial Q}{\partial d_0}\right]^2\delta d_0^2 + \left[\frac{\partial Q}{\partial d_1}\right]^2\delta d_1^2$$
	$$\delta Q = \sqrt{{\left(\frac{\pi \,C_{d}\,d_{0}\,\sqrt{\frac{2(\Delta h)\,g}{1-\frac{{d_{0}}^4}{{d_{1}}^4}}}}{2}+\frac{\pi \,\sqrt{2}\,C_{d}\,(\Delta h)\,{d_{0}}^5\,g}{2\,{d_{1}}^4\,{\left(\frac{{d_{0}}^4}{{d_{1}}^4}-1\right)}^2\,\sqrt{\frac{(\Delta h)\,g}{1-\frac{{d_{0}}^4}{{d_{1}}^4}}}}\right)}^2\,{\delta d_{0}}^2-\left(\frac{{C_{d}}^2\,(\Delta h)\,{d_{0}}^{12}\,g\,\pi ^2}{2\,{d_{1}}^{10}\,{\left(\frac{{d_{0}}^4}{{d_{1}}^4}-1\right)}^3}\right)\,{\delta d_{1}}^2}$$
	\begin{equation}
		\rightarrow \delta Q = \frac{\pi \,\sqrt{2}}{2}\,\sqrt{-\frac{{C_{d}}^2\,(\Delta h)\,{d_{0}}^2\,{d_{1}}^2\,g\,\left({d_{0}}^{10}\,{\delta d_{1}}^2+{d_{1}}^{10}\,{\delta d_{0}}^2\right)}{{\left({d_{0}}^4-{d_{1}}^4\right)}^3}}
	\end{equation}
	where it is given that $\delta d_0=\delta d_1=0.2mm=2\times10^{-4}m$ for both the Orifice and Venturi tube diameter measurements.
	\subparagraph{Relative Difference:}
	To evaluate the calculated volumetric flow rates, a relative difference was computed for each. For some arbitrary volumetric flow rate computed under a scenario X, the difference relative to the measured flow rate, $Q_m$, was calculated as:
	\begin{equation}
		Q_{mX}=\frac{Q_m - Q_X}{Q_X}\times100\%
	\end{equation}
	
\end{document}